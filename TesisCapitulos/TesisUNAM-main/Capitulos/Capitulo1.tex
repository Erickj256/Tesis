\chapter{\textbf{Modelo SIR}}

El modelo SIR es un modelo compartimental determinista que es utilizado con bastante frecuencia en el área de la epidemiología. Los modelo compartimentales se definen como la segmentación de nuestra población en conjuntos o compartimentos disjuntos que describen el estado de salud en el que se encuentra una población ante la propagación de una enfermedad, este tipo de modelos no contempla la aleatoriedad.

El modelo SIR es un modelo compartimental determinista que describe la dinámica de la población al contraer una enfermedad a través de un sistema de ecuaciones diferenciales el cual es representado con el siguiente diagrama de flujo.

\begin{figure}[h]
\centering
\begin{tikzpicture}[node distance=2cm, auto, >=Latex, 
every node/.append style={align=center},int/.style={draw, minimum 			size=3cm}]
\node [int](S) {$S$};
\node [int, right=of S](I) {$I$};
\node [int, right=of I](R) {$R$};
\coordinate[right=of I] (out);
\path[->] (S) edge node {$\beta$} (I)
   			  (I) edge node {$\gamma$} (R);
\end{tikzpicture}
\caption{Modelo SIR} \label{fig:1.1}
\end{figure}	

Donde cada una de las cajas representa un compartimento que describe el estado de salud de la población, mientras que las flechas representan el movimiento de los individuos a traves de los tres compartimentos que en este modelo son:

\begin{itemize}
\item Suceptible(S): Representa el número de personas suceptibles, personas sanas que al entrar en contacto con la enfermedad pueden contraer la infección.

\item Infectado (I): Representa el número de personas infectadas, personas que adquirieron la enfermedad y puden transmitir la enfermedad.

\item Recuperado (R): Representa el número de personas recuperadas, personas que se han recuperado en su totalidad de la enfermedad y puden o no generar inmunidad.

\item \textbf{$\beta:$} Tasa de contagio

\item \textbf{$\gamma:$} Tasa de recuperación

\end{itemize}

Antes de establecer el sistema de EDO's que representan al modelo SIR es necesario establecer las siguientes suposiciones.

\begin{itemize}
\item {El estado inicial de la población es suceptible}
\item {La enfermedad es contagiada de forma directa}
\item {La población es constante, esto indica que no se consideran los nacimientos ni las defunciones que se producen a lo largo de la enfermedad por tanto
	   si, denotamos como N al total de la población tendremos que 
	   \begin{equation} \label{eq1}
	   N = S + I + R
	   \end{equation}
	   }
\item {La población es cerrada, esto nos indica que no se permite la salida ni el ingreso de personas a la población}
\item {La población recuperada no pueden reinfectarse provocando que el movimiento de recuperados a suceptibles es nulo}
\end{itemize}

mientras que su sistema de ecuaciones diferenciales ordinarias que lo representa, esta definido de la siguiente manera:


\begin{numcases}{ }
S' = -\beta S\left(t\right)I\left(t\right) \label{eq1.2}\\			
I' = \beta S\left(t\right)I\left(t\right) - \gamma I\left(t\right) \label{eq1.3}\\		
R' = \gamma I\left(t\right) \label{eq1.4}
\end{numcases}


\section{Resultados del modelo SIR}
\subsection{Existencia de limites en el modelo SIR}
Considerando las suposiciones mencionadas anteriormente podemos notar que existen limites a lo largo de la enfermedad,(\ref{eq1.2}) es decreciente así $S' \leq 0$ y por tanto $0 \leq S(0) \leq S(t) \leq N$, (\ref{eq1.4}) es creciente por lo que $R' \geq 0$ así $0 \leq R(0) \leq R(t) \leq N$ \cite{Foundations}.

Con esto podemos establecer las siguientes afirmaciones:

\begin{Af}
$\lim_{t \to \infty} S(t) = S_{\infty}$
\end{Af}

\begin{Dem}
Notemos que  $0 \leq S(0) \leq S(t) \leq N$ por tanto $S(t)$ esta acotada inferiormente, al ser acotada inferiormente en $t \in [0, \infty)$ podemos considerar a $S_{\infty}$ como infimo, por lo que
 
\begin{align*}
\forall \epsilon >0  \exists \hat{t} > 0 : S_{\infty} - \epsilon \leq S(\hat{t}) \leq S_{\infty} +  \epsilon
\end{align*}
 
Al ser S(t) decreciente 

\begin{align*}
S_{\infty} - \epsilon \leq S(t) \leq S(\hat{t}) \leq S_{\infty} +  \epsilon  \hspace{1cm} \forall t  > \hat{t} 
\end{align*}

Por tanto 

\begin{align*}
\forall \epsilon > 0  \exists \hat{t} > 0 : |S(t) -  S_{\infty}| <  \epsilon \hspace{1cm} \forall t  > \hat{t} 
\end{align*}

Así entonces

\begin{align*}
\lim_{t \to \infty} S(t) = S_{\infty}
\end{align*}

\end{Dem}

\hfill	$\square$

\begin{Af}
$\lim_{t \to \infty} R(t) = R_{\infty}$
\end{Af}

\begin{Dem}
Notemos que al igual que S(t), $0 \leq R(0) \leq R(t) \leq N$ por tanto  $R(t)$ esta acotada superiormente, al ser acotada superiormente en $t \in [0, \infty)$ podemos considerar a $R_{\infty}$ como su supremo, por lo que
 
 \begin{align*}
\forall \epsilon >0  \exists \hat{t} > 0 : R_{\infty} - \epsilon \leq R(\hat{t}) \leq R_{\infty} +  \epsilon
 \end{align*}
 
Al ser R(t) creciente 

\begin{align*}
R_{\infty} - \epsilon \leq R(\hat{t}) \leq R(t) \leq R_{\infty} +  \epsilon  \hspace{1cm} \forall t  > \hat{t} 
\end{align*}

Por tanto 

\begin{align*}
\forall \epsilon > 0  \exists \hat{t} > 0 : |R(t) -  R_{\infty}| <  \epsilon \hspace{1cm} \forall t  > \hat{t} 
\end{align*}

Así entonces

\begin{align*}
\lim_{t \to \infty} R(t) = R_{\infty}
\end{align*}

\end{Dem}

\hfill	$\square$

\begin{Af}
$\lim_{t \to \infty} I(t) = I_{\infty} = N - S_{\infty} - R_{\infty}$
\end{Af}

\begin{Dem}

Considerando que $I (t) = N - S(t) - R(t)$ se tiene que: 

\begin{align*}
\lim_{t \to \infty} I(t)  = \lim_{t \to \infty}  N - S (t) - R(t) 
\end{align*}

además tomando $\lim_{t \to \infty} R(t) = R_{\infty}$ y $\lim_{t \to \infty} S(t) = S{\infty}$ podemos concluir que 

\begin{align*}
 I_{\infty}  = N - S_{\infty} - R_{\infty}
\end{align*}

\end{Dem}

\hfill	$\square$

De este último resultado, hay que recordar que la población infectada se moverá al compartimento de los recuperados en un determinado tiempo y como consecuencia nuestro compartimento
de infectados ya no tendra población estableciendo así la siguiente afirmación.

\begin{Af}
Las infecciones tienden a desaparecer es decir $I_{\infty} = 0$
\end{Af}

\begin{Dem}

Considerando $\lim_{t \to \infty} R'(t)$ tenemos que

\begin{align*}
\lim_{t \to \infty} R'(t) = \lim_{t \to \infty}  I (t) \gamma = \gamma I_{\infty}
\end{align*}

Entonces 

\begin{align*}
\forall \epsilon, \exists \hat{t} > 0 : |R'(t) - \gamma	I_{\infty}| < \epsilon 
\end{align*}

Considerando $\epsilon = \frac{\gamma	I_{\infty}}{2}$

\begin{align*}
 - \frac{\gamma	I_{\infty}}{2} < R'(t) - \gamma	I_{\infty} < \frac{\gamma	I_{\infty}}{2}
\end{align*}

Sumando  $\gamma I_{\infty}$ tenemos
\begin{align*}
R'(t) > \frac{\gamma I_{\infty}}{2} \hspace{1 cm} \forall t > \hat{t}
\end{align*}

Por ultimo integramos 
\begin{align*}
R(t) > \frac{\gamma I_{\infty}}{2} t  + K
\end{align*}

Lo cual contradice el hecho de que R(t) es acotado, por tanto $I_{\infty} = 0$

\end{Dem}

\hfill	$\square$

\subsection{Afirmaciones del modelo SIR}
Con el transcurso de la enfermedad se plantean nuevas preguntas que de cierta forma son naturales y encontrar la respuesta a estas no ayudaría a comprender el cómo transcurre la enfermedad de una forma aproximada y así poder tomar mejores decisiones para afrontarla. Estas preguntas son : ¿Existe un número máximo de infectados?, ¿Cuál es el tamaño final de la pandemia?, ¿Se tiene un resultado que nos permita saber el número de infecciones que causa un individuo? y por último ¿Existe un resultado que nos permita conocer la tendencia del brote?.

Buscaremos dar respuesta a las preguntas planteadas anteriormente, pues conocer la respuesta es de suma importancia para plantear posibles soluciones al brote teniendo así las siguientes afirmaciones.


\begin{Af}
Número máximo de infectados esta dado por $I_{max} = - S(t) + \frac{\gamma}{\beta} ln(S(t)) + I(0) + S(0) - \frac{\gamma}{\beta} ln(S(0))$
\end{Af}

\begin{Dem}

Tomando el cociente de las ecuaciones (\ref{eq1.2} y \ref{eq1.3})se tiene lo siguiente.

\begin{align*}
\frac{dI}{dS} = - \frac{\beta SI - \gamma I}{\beta SI} = ...
\end{align*}

\begin{align*}
... = -1 + \frac{\gamma}{\beta S}
\end{align*}

resolviendo por variables separables obtenemos

\begin{align*}
\int I = \int (-1 + \frac{\gamma}{\beta S} dS = ...
\end{align*}

\begin{align*}
... I = - S + \frac{\gamma}{\beta} ln(S) + C
\end{align*}

Obteniendo que  $C = I(0) + S(0)  - \frac{\gamma}{\beta} ln(S(0))$ considerando condiciones iniciales 

\begin{align*}
I_{max} = - S(t) + \frac{\gamma}{\beta} ln(S(t)) + I(0) + S(0) - \frac{\gamma}{\beta} ln(S(0))
\end{align*}

y cómo $I' = 0$  si  $S = \frac{\gamma}{\beta}$ 

\begin{align*}
I_{max} = \frac{\gamma}{\beta} + \frac{\gamma}{\beta} ln(\frac{\gamma}{\beta}) + I(0) + S(0) - \frac{\gamma}{\beta} ln(S(0))
\end{align*}


\end{Dem}

\hfill	$\square$

\begin{Af}
El tamaño final de la pandemia esta dado por $R(\infty) = N - S_{\infty}$
\end{Af}

\begin{Dem}

Usando (\ref{eq1.2}) y (\ref{eq1.4}) además de considerar que $S' \leq 0$ y $0 \leq R'$ obtenemos 

\begin{align*}
\frac{dS}{dR} = \frac{-\beta S}{\gamma}
\end{align*}

resolviendo por variables separables se tiene:

\begin{align*}
\int \frac{1}{S} dS = \int \frac{-\beta}{\gamma} dR 
\end{align*}

\begin{align*}
\ln(S) = \frac{-\beta}{\gamma} R + C
\end{align*}

Considerando las condiciones iniciales de S y recordando que $0 \leq R(0) \leq R(t) \leq N$

\begin{align*}
S = S(0) e^{\frac{-\beta}{\gamma}R}
\end{align*}

\begin{align*}
S = S(0) e^{\frac{-\beta}{\gamma}R} \geq S(0) e^{\frac{-\beta}{\gamma}N} > 0
\end{align*}

\begin{align*}
R(\infty) = N - S_{\infty}
\end{align*}

\end{Dem}

\hfill	$\square$

Con el transcurso de la pandemia el número de población suceptible disminuye y como consecuencia el ritmo de las nuevas infecciones disminuye es decir, $S(t)$ cae por debajo de $\frac{\beta}{\gamma}$ y la tasa de recuperación excede a la tasa de infección provocando que $I(t)$ disminuya. Esto siginifica que la pandemia termina por la falta de nuevos infectados y no por la falta de personas suceptibles.

\subsection{Número reproductivo basico,  Tasa de reproducción efectiva  y Teorema del umbral}
A la expresión $\frac{\beta}{\gamma}$ se le conoce como número reproductivo basico y se define como:

\begin{align}
R_{0} = \frac{\beta}{\gamma}
\end{align}

El cual es el promedio de infecciones causadas por un individuo enfermo en una población durante el tiempo en el que este individuo permanece infeccioso, partir de $R_{0}$ podemos establecer $R_{e}$ el cual es conocido como la Tasa de reproducción efectiva y se definede la siguiente manera  

\begin{align}
R_{e} (t) = S(t)R_{0} 
\end{align}

La diferencia entre los valores $R_{0}$ y  $R_{e}$, radica en que $R_{0}$ es un valor fijo y ayuda a describir la propagación inicial de la enfermedad en una población completamente suceptible, mientras que $R_{e}$ suele ser dinamico pues refleja la propagación de la enfermedad considerando los cambios de la población suceptible. \cite{Brauer}

una vez estableciendo estos valores, podemos establcer el Teorema del umbral \cite{Foundations}, este teorema es de suma importancia pues establece la existecia de un valor critico del número de individuos suceptibles en una población a partir del cual se podra determinar la ocurrencia de una epidemia o no

\begin{teo}
Teorema del umbral.

Si  $R_{e} < 1$ entonces I(t) decrece y no ocurre una epidemia pues la enfermedad sede 

Si  $R_{e} > 1$ entonces I(t) crece rapidamente probocando una epidemia. 
\end{teo}

\begin{Dem}

Consideremos  el caso cuando  $R_{e} < 1$, si partimos de {\ref{eq1.3}} 

\begin{align*}
I(t) (\beta S(t) - \gamma)  <  0 
\end{align*}

ya que  $I'(t) > 0$ y además $I'(t)$ es decreciente  entonces  $(\beta S(t)  - \gamma) < 0$ 

\begin{align*}
\beta S(t)  < \gamma 
\end{align*}

\begin{align*}
S(t)  < \frac{\gamma}{\beta} 
\end{align*}

\begin{align*}
R_{0} \cdot S(t)  < \frac{1}{R_{0}} \cdot R_{0}  \hspace{1cm} R_{0} > 0
\end{align*}

\begin{align*}
R_{e}  < 1
\end{align*}

Cuando $R_{e}  < 1$ nos indica que cada individuo infectado genera en promedio menos de un caso nuevo, por lo que la pandemia no puede sostenerse ya que el número de infectados disminuye con el tiempo. 


Consideremos  el caso cuando  $R_{e} > 1$, nuevamente partiendo de {\ref{eq1.3}} 

\begin{align*}
I(t) (\beta S(t) - \gamma)  > 0 
\end{align*}

\begin{align*}
\beta S(t)  > \gamma 
\end{align*}

\begin{align*}
S(t)  > \frac{\gamma}{\beta} 
\end{align*}

\begin{align*}
R_{0} \cdot S(t)  > \frac{1}{R_{0}} \cdot R_{0} \hspace{1cm} R_{0} > 0
\end{align*}

\begin{align*}
R_{e}  > 1
\end{align*}

Cuando $R_{e}  > 1$ nos indica que cada individuo infectado genera en promedio más de un caso nuevo, por lo que la pandemia comienza a crecer con el tiempo. 

\end{Dem}

\hfill	$\square$

\section{Parámetros}
\subsection{Tasa de recuperación \textbf{$\gamma$}}

Una parte fundamental de los modelos compartimentales son sus parámetros en este caso $\beta$ (tasa de contagio) y $\gamma$(tasa de recuperación), ya que estos parametros son los que nos proporcionaran la información sobre el comportamiento de la enfermedad que buscamos modelar \cite{Martcheva}.

Para poder encontrar un valor aproximado al parámetro $\gamma$, asumiremos que no se estan generando nuevas personas infectadas, por lo que el flujo de personas que entra a este compartimento ha parado obteniendo la siguiente ecuación.

\begin{align}
I' = -\gamma I, \hspace{1cm} I(0) = I_{0}
\end{align}

\begin{Af}

El número de personas en el compartimento de los infectados esta dado por $\frac{I(t)}{I_{0}} = e^{-\gamma t}$.

\end{Af}

\begin{Dem}
En efecto, al resolver la ecuacion 1.8 por variables separables, tenemos: 

\begin{align*}
\frac{dI}{dt} = - \gamma I
\end{align*}

\begin{align*}
\ln{I} = - \gamma t + C
\end{align*}

\begin{align*}
I = e^{- \gamma t} \cdot e^{C}
\end{align*}

\begin{align*}
I = I_{0} e^{- \gamma t}
\end{align*}

\begin{align*}
\frac{I(t)}{I_{0}} = e^{-\gamma t}
\end{align*}

\hfill	$\square$

\end{Dem}

Con estas ecuaciones previamente obtenidas podemos calcular de forma probabilística la cantidad de personas que han abandonado el compartimento de los infectados.
\begin{align}
F(t) = 1 -  e^{-\gamma t} \hspace{1cm} t \geq  0
\end{align}

Definiendo la función de desidad en el intervalo [0,t) de la siguiente forma.

\begin{equation*}
F(t) =
\begin{cases*}
\gamma e^{-\gamma t} \hspace{1cm} t \geq  0 \\
0 \hspace{1.7cm} t <  0
\end{cases*}
\end{equation*}

y dado que el tiempo promedio que pasa una persona en el compartimento de los infectados está dado por valor el esperado, así obtenemos que el parametro $\gamma$ se encuentra definido como:

\begin{align}
E[X] = \int \limits_{0}^{\infty} tf(t)dt = \int \limits_{0}^{\infty} t\gamma e^{-\gamma t} dt = \frac{1}{\gamma}
\end{align}

\begin{Af}
El tiempo de permanencia de una persona en la clase I es $\frac{1}{\gamma}$. \cite{Martcheva}
\end{Af}

\begin{Dem}

Para mostrar que en efecto $\frac{1}{\gamma}$ es el tiempo que permanece una persona en la clase I, basta con resolver la integral impropia.

\begin{align*} 
\int \limits_{0}^{\infty} t\gamma e^{-\gamma t} dt = \lim\limits_{b\rightarrow \infty} \int \limits_{0}^{b} t\gamma e^{-\gamma t}dt 
\end{align*}

\begin{align*}
\lim\limits_{b\rightarrow \infty} \int \limits_{0}^{b} t\gamma e^{-\gamma t}dt = \lim\limits_{b\rightarrow \infty} \left[ \gamma \cdot \left( \frac{-t e^{-\gamma t}}{\gamma} - \int \limits_{0}^{b} \frac{-e^{-\gamma t}}{\gamma} dt \right)\right] = \cdots
\end{align*}

\begin{align*}
\cdots = \lim\limits_{b\rightarrow \infty} \left( \left[ -te^{- \gamma t} - \frac{e^{-\gamma t}}{\gamma} \right]_0^b \right) 
\end{align*}

\begin{align*}
\cdots = \lim\limits_{b\rightarrow \infty} \left[ -be^{- \gamma b} - \frac{e^{- \gamma b}}{\gamma} \right] - \lim\limits_{b\rightarrow \infty} \left[ -(0)e^{- \gamma (0)} - \frac{e^{- \gamma (0)}}{\gamma} \right] = \frac{1}{\gamma}  \\
\end{align*}

\hfill	$\square$

\end{Dem}

Podemos concluir que $\gamma$ \footnote{El resultado de la ecuación 1.10 lo obtenemos en termino de gamma, para no causar confucion dicha gamma se cambia por d, donde d representa los días que tarda una persona en recuperarse de la enfermedad} está definida con el inverso de los días que una persona tarda en recuperarse de la infección, por tanto tenemos que $\gamma = \frac{1}{d}$.

Con el resultado previo y utilizando la ecuacion 1.7, podemos establecer la siguiente afirmación.

\begin{Af}
La cantidad de personas dentro de la clase R es $R(t+\frac{1}{\gamma}) = R(t) + I(t)$
\end{Af}

\begin{Dem}
\begin{align*}
\gamma I = R'(t)
\end{align*}

\begin{align*}
\gamma I = R'(t) = \frac{R\left( t + h \right) - R\left(t\right)}{h}, \hspace{1cm} h \neq 0
\end{align*}

tomando $h = \frac{1}{\gamma}$ y despejando: 

\begin{align*}
\gamma I = R'(t) = \frac{R\left( t + \frac{1}{\gamma} \right) - R\left(t\right)}{\frac{1}{\gamma}}
\end{align*}

por tanto

\begin{align*}
R\left( t + \frac{1}{\gamma} \right) = R(t) + I(t)
\end{align*}

\hfill	$\square$
\end{Dem}

Con este último resultado podemos interpretar que el número de individuos recuperados en un determinado tiempo $t + \frac {1}{\gamma}$  es la suma de las personas que se recuperarón en un tiempo t más las personas infectadas en el tiempo t.

\subsection{\textbf{Tasa de contagio}}

\begin{Af}
Podemos aproximar al parametro $\beta$ a partir de $ln(I) = mt + ln(I_{0})$\cite{Algeria}
\end{Af}

\begin{Dem}
Partiendo de la ecuación 1.6 
\begin{align*}
\frac{dI}{dt} = \beta SI - \gamma I
\end{align*}

factorizando I y tomando $m = \left(\beta S - \gamma \right)$

\begin{align*}
\frac{dI}{dt} = I \left( \beta S - \gamma \right)
\end{align*}

\begin{align*}
\frac{dI}{dt} = mI
\end{align*}

resolviendo la ecuación diferencial anterior por variables separables, se tiene:

\begin{align*}
ln(I) = mt + c
\end{align*}

\begin{align*}
I = I_{0} e^{mt} 
\end{align*}

Aplicando logaritmo:

\begin{align*}
ln(I) = mt + ln(I_{0})
\end{align*}

\hfill	$\square$

\end{Dem}


%\begin{thebibliography}{4}

%\bibitem{Brauer}
%Brauer, F., Castillo-Chavez, C., Feng, Z. (2019). Mathematical Models in Epidemiology. Alemania: Springer New York.

%\bibitem{Martcheva}
%Martcheva, M. (2015). An Introduction to Mathematical Epidemiology. Alemania: Springer US.

%\bibitem{enfInf}
%Montesinos-López, Osval Antonio, \& Hernández-Suárez, Carlos Moisés. (2007). \textit{Modelos matemáticos para enfermedades infecciosas}. Salud Pública de México, 49(3), 218-226.

%\bibitem{Algeria}
%Lounis M, Bagal DK. Estimation of SIR model's parameters of COVID-19 in Algeria. Bull Natl Res Cent. 2020,44(1):180.



%\end{thebibliography}
%\end{document}