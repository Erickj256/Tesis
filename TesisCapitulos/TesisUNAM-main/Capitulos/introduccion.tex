%\documentclass[a4paper,openany,12pt]{book}
%\usepackage[utf8]{inputenc}
%\usepackage[spanish,es-noshorthands]{babel}
%\usepackage{amsthm}
%\usepackage{amsmath}
%\usepackage{amsfonts}
%\usepackage{amssymb}
%\usepackage{tikz}
%\usepackage{caption}
%\usepackage{natbib}
%\usetikzlibrary{arrows.meta,positioning}
%\newtheorem*{Introducción}{Introducción}
%\newtheorem*{Definición}{Definición}
%\newtheorem*{Lema}{Lema}
%\newtheorem*{Teorema}{Teorema}
%\usepackage{ragged2e}
%\usepackage{graphicx}
%\usepackage[left=2.5cm,right=2.5cm,top=2.5cm,bottom=2.5cm]{geometry}
%\setlength{\parindent}{0mm}

%\begin{document}

\chapter*{Introducción}

Las enfermedades infecciosas han acompañado a los seres vivos desde que la agricultura sedentarizo a los diversos grupos nómadas y les hizo formar poblaciones, con esto los factores de infección se multiplicaron ya que aumentaron las interacciones entre distintos microorganismos (virus y bacterias).\\

Cuando se produce este contacto entre el organismo hospedante y el microrganismo el cual es capaz de sobrevivir y reproducirse en su interior, inicia entre ambos un proceso de adaptación que en ocasiones puede provocar la muerte del organismo hospedante, de esta forma, cuando varios individuos contraen el microorganismo a la vez se comienza a denominar como epidemia y si esta escala de forma global recibe el nombre de pandemia\\

Estas infecciones o enfermedades no son debilidades ni defectos biológicos ya que su existencia es un factor necesario e indispensable para el ciclo de la vida, ya que la interacción entre las especies y los microorganismos generan una disminución de la población, forzando así una adaptación en los seres vivos.\\

A pesar de las adaptaciones que han sufrido los individuos a través del tiempo, estos no están exentos de volver a interactuar con nuevos microorganismos. Esto como resultado de las modificaciones, así como a las interacciones con el medio ambiente y sus elementos. \cite{HyFP}\\

%Durante este tiempo se han registrado eventos de esta naturaleza donde algunos de los ejemplos más relevantes han sido:
%La peste negra (1346), La gripe española (1918), El SARS (2002)\\

\newpage
Una vez que el ser humano experimento estas enfermedades, comenzó a desarrollar métodos o técnicas para el estudio de estas enfermedades 
como lo es la modelación matemática el cual tiene como principal objetivo el comprender los fenómenos naturales para extraer la información necesaria y describir el fenómeno de interés matemáticamente.\\ 

Uno de los primeros modelos matemáticos que se realizaron para entender una enfermedad, fue el que realizo Daniel Bernoulli durante el brote de la viruela que fue altamente contagiosa, el segundo modelo pertenece a Ronald Ross quien explico el ciclo completo de la malaria humana, los siguientes en desarrollar un modelo fueron W.O Kermack y A.G McKendrick también conocido como el modelo SIR, este modelo presenta un resultado importante conocido como el Teorema Umbral el cual nos permite conocer más acerca del brote de la enfermedad.\\

A finales del año 2019, se presentó una nueva enfermedad la cual carecía de una sintomatología exacta, esta enfermedad se identificó por primera vez en la región de Wuhan China la cual se identifico como el SARS-COV2 o COVID 19 por la OMS y que tuvo un fuerte impacto tanto en la población China como en el resto del mundo, esto debido a la globalización y a la interconexión que existe entre los países.\\

Gracias a los modelos desarrollados anterior y principalmente al modelo SIR permitieron establecer condiciones para entender y comprender los posibles escenarios del desarrollo de la enfermedad en la población y su posible duración.\\

 

%\begin{thebibliography}{3}

%\bibitem{Gpysd}
%Castañeda Guillot, Carlos, Martínez Martínez, Ronelsys, `I\&' López Falcón, Adriana. (2021). \textit{pandemias y sus desafíos}. Dilemas contemporáneos: %educación, política y valores, 8(3), 00047. Epub 11 de junio de 2021. https://doi.org/10.46377/dilemas.v8i3.2671

%\bibitem{HyFP}
%Sanchez-Gonzales Miguel A. \textit{Historia y futuro de las pandemias}, REV. MED. CLIN. CONDES - 2021; 32(1) 7-13. 
%DOI: 10.1016/j.rmclc.2020.12.007

%\end{thebibliography}

%\end{document}