%\chapter*{Introducción}

Las enfermedades infecciosas han acompañado a los seres vivos desde que la agricultura sedentarizo a los diversos grupos nómadas y les hizo formar poblaciones, con esto los factores de infección se multiplicaron ya que aumentaron las interacciones entre distintos microorganismos (virus y bacterias).

Cuando se produce este contacto entre el organismo hospedante y el microrganismo el cual es capaz de sobrevivir y reproducirse en su interior, inicia entre ambos un proceso de adaptación que en ocasiones puede provocar la muerte del organismo hospedante, de esta forma, cuando varios individuos contraen el microorganismo a la vez se comienza a denominar como epidemia y si esta escala de forma global recibe el nombre de pandemia.

Estas infecciones o enfermedades no son debilidades ni defectos biológicos ya que su existencia es un factor necesario e indispensable para el ciclo de la vida, ya que la interacción entre las especies y los microorganismos generan una disminución de la población, forzando así una adaptación en los seres vivos.

A pesar de las adaptaciones que han sufrido los individuos a través del tiempo, estos no están exentos de volver a interactuar con nuevos microorganismos pues el medio ambiente y sus elementos sufren modificaciones día a día debido a las interacciones que tiene con los individuos. %\cite{HyFP} revisar

Conforme el ser humano fué experimentando estas enfermedades comenzó a desarrollar métodos o técnicas para su estudio, uno de estos métodos es la modelación matemática, la cuál tiene como principal objetivo el comprender los fenómenos naturales para extraer la información necesaria y describir el fenómeno matemáticamente. La modelación matemática no permitió establecer modelos que nos ayudaron a comprender una infección, uno de los primeros modelos fué el establecido por Daniel Bernoulli durante el brote de la viruela en (fecha), el segundo modelo establecido por Ronald Ross explico el ciclo completo de la malaria humana, los siguientes en desarrollar un modelo fueron W.O Kermack y A.G McKendrick y es conocido como el modelo SIR, este modelo presenta un resultado importante conocido como el Teorema Umbral que nos permite conocer más acerca del flujo de la infección.

A finales del año 2019, se presentó una nueva enfermedad la cual carecía de una sintomatología exacta, esta enfermedad se identificó por primera vez en la región de Wuhan China a la cual se nombro debido a sus caracteristicas como SARS-COV2 o COVID 19 por la OMS y que tuvo un fuerte impacto tanto en la población China como en el resto del mundo, esto debido a la globalización y a la interconexión que existe entre los países.

Gracias a los modelos desarrollados anteriormente y principalmente al modelo SIR permitieron establecer condiciones para entender y comprender los posibles escenarios del desarrollo de la enfermedad en la población y su posible duración.
